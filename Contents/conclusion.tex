\chapter{总结与展望}
基于位置服务的需求使得室内定位系统IPS有很大的潜力,可见光定位VLP技术凭借其低成本、相对较高的定位精度在各种的室内定位技术中有很大的优越性,受到越来越多的关注。然而,从鲁棒性的角度来说,由于光沿直线传播LOS的特性,导致VLP在面临遮挡或者阴影的时候表现不佳。本文的主要内容就是考虑在面临遮挡的情况下如何使VLP正常运行。基于此,本文提出利用一次反射光进行VLP以解决LOS受阻的问题。

除了遮挡之外,很多的关于VLP的演示系统过于复杂,难以在实际环境中进行推广。比如基于指纹匹配的定位算法需要前期花费大量人力物力建设和维护指纹库,TOA算法需要较高的时间同步准确性。更为关键是,基本上所有的VLP算法都需要接收端同时捕捉大量的LED,然而接收端有限的视场角限制了这一要求。因此,如何使VLP算法在较少数量的LED的情况下依旧能够正常运行,并且尽量使系统更为简单易于推广也是本文要解决的一个难点。

基于此,本文提出了使用单个LED和图像传感器IS来进行非视距NLOS IS-VLP。图像传感器有两个比较大的优势,一方面,相机的普及程度非常高;另外一方面,IS对于NLOS场景下低信噪比信号的抗干扰能力更强。因此,相较于光电探测器PD,本文选择了IS作为接收端,一方面可以使定位系统性能更好,另外一方面使系统更加易于推广。

就系统性能来说,NLOS IS-VLP解决了传统LOS VLP面临的遮挡问题,并且在只有单个LED的情况下实现较高的定位精度。然而,不可忽视的一个问题是,这种基于一次反射光的NLOS IS-VLP系统,对反射面材质和粗糙程度都有一定的要求。当反射面的条件不佳时,系统性能可能衰减厉害。

就本文工作的主要内容来说,在一些方面依然面临一些挑战有待解决。首先,对于NLOS OCC系统来说,LED的非线性和电容效应以及反射光的低信噪比增加了信号解调的难度。利用非线性优化算法对采样信号进行矫正是值得尝试的一种思路。

其次,目前IS-VLP系统,需要不断切换长短曝光来获取明暗条纹和捕捉参考点的像素坐标,这种在录制的过程中快速切换曝光时间的机制很难实际实现。另外,参考点的检测识别带来的像素误差也是影响定位精度的主要原因。因此,如何在适当的曝光时间下,仅通过一帧图像同时实现提取明暗条纹和检测参考点的坐标是一个非常值得挑战的难题。针对这个问题,通过深度学习模型进行图像处理是一个值得借鉴的思路。

最后,针对本文所提出的两种NLOS IS-VLP方案,依旧有一些可以提升的地方。一方面,基于亮度分布模型的NLOS IS-VLP系统通过建立一种光学成像模型将基于单个LED的NLOS IS-VLP转化为基于两个虚拟LED的LOS IS-VLP,系统只能在相机成像平面平行于地面的情况下才能工作,并且定位精度有待提高。因此,接下来可以考虑增加姿态角传感器来实现任意姿态的定位。另外一方面,基于双目立体视觉的NLOS IS-VLP系统,系统存在的主要问题除了反射面不理想带来的性能衰退之外,还有就是IMU姿态角度估计的准确性以及双目相机Z轴误差较大等问题。使用滤波算法或者强化学习以及循环神经网络等机器学习方法对系统的局部或者全局进行优化是解决思路之一。

针对VLP系统的未来发展,首先依然是要更完美的解决本文提出的两个主要挑战。

从系统设计的角度来说,基于单一信源特别是有着直线传播特性的VLP,复杂环境下很难实现可靠的定位。因此,基于多源融合的IPS是很有必要的,特别是射频RF信号的穿透性,是不可或缺的。综合考虑定位精度、成本以及稳定性,通过融合RF信源和可见光的IPS将会是未来IPS的发展方向。这样的融合定位系统,可以根据不同的室内环境动态的调整多源定位的权重,在保证系统稳定性的同时,实现低成本和多数情况下的高精度定位。

从硬件系统不断发展的角度来说,针对LED数量的问题,mini-LED阵列可以作为发射端替代原来的LED使得同样的区域信标数量大大增加。针对LOS遮挡以及非理想反射面导致的一次反射光信噪比太低问题,使用高灵敏度的探测器如单光子探测器SPAD等是值得研究的。


